\section{Letter}
Dear readers,

In March 2014, the Ministry of Health of Guinea reported an outbreak of Ebola virus disease (EVD) in four southeastern districts: Guekedou, Macenta, Nzerekore and Kissidougou. A total of 86 suspected cases, including 59 deaths was reported as of March 24. Since then, the spread of EVD in West Africa has explosively grew. According to the latest statistics, EVD has caused 13855 cases and 9004 deaths. Faced with the havoc, the world have expressed their deep concern. In the last year, a mess of aid and donations converged on West Africa, medical orgnizations all over the world also dedicated to the researh of medication for EVD. With respect to people's determination of eradicating EVD, we now present our new progress on medication research.


The drug we developed aiming at EVD has been proved to have an amazing curative effect for the patients whose main organ hasn't been damaged. According to the data on clinical trial, 18 of 23 volunteers were cured, with an average recovery time of 8.4 days. It is gratifying that, with goverment's ratification, the drug has been put into mass production now. So patients will be able to receive effective medical treatment before long. The corresponding vaccine has also been developed, which is still in test stage. Vaccination against EVD can be expected soon.


However, as much as we might wish it, we can't provide the drug to all the patients without enough production capacity. To minimize the damage West Africa suffered, our only choice is to set cities' priority in drug delivery according to its population and location. Outbreak in populous cities is more destructive and uncontrollable, while cities near the epidemic focus stand a great chance of breaking out EVD. These cities will be firstly considered in our delivery system. Transportation hubs also have a high priority in our system as EVD may spread through transportation networks. Glacial as it sounds, this method has the highest efficiency in current situation. The latest arrangement details of medication delivery can be checked on our official website.

Though we are still in lack of effective medication, it is worth emphasizing that the confrontation against EVD needs not only medical researches, but also efforts of every single individual. A lot can be done to protect yourselves as well as others from infection. Some useful tips are given as follow

\begin{itemize}
	\item \textbf{Scale back on going out}
	      the transmission of EVD requires direct contact with the infected or their body fluids and blood. Reducing the contact with others is a proper way of avoiding infection.
	\item \textbf{Maintain personal hygiene}
	      This tip goes without saying. Personal hygiene is always of great significance even no epidemic disease is spreading. Maintain personal hygiene also means do not go to public places with appalling sanitary conditions.
	\item \textbf{Check data updates}
	      Keep focused on the data updates of EVD. This can help you get better acquainted with the situation and be prepared.
	\item \textbf{Be cautious}
	      If you feel uncomfortable, go to the medical institution nearby as soon as possible. You can suggest your friends to do the same. This act may prevent lives from fading away.
\end{itemize}

If you are unfortunately infected with EVD, don't panic. EVD is not absolute lethal. You can still stand a good chance of recovery as long as doctor's advices are followed.

The status of fighting against EVD remains severe now, and we will continue our efforts in medication development and medical assistance. To our belief, EVD will be thoroughly eradicated in the near future.