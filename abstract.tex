\begin{abstract}
	The first recorded Ebola outbreak occurred in Nzara, South Sudan in 1967 which infected 284 and killed 151. Since then, Ebola outbreak occurs frequently and there are nearly 20 cases of major or massive outbreaks till now, the most recent of which is the case in Liberia, Guinea and Sierra Leone in 2014. The case fatality rate ranges from 25\%(Uganda, 2007) to 90\%(Republic of Congo, 2002). Considering its high fatality rate and damage to human society, it is highly valuable to study the property of the spread of Ebola and to find out feasible strategy to fight against the virus.

	In this paper, we attempted to untangle the convolution of parameters and variables concerning the spread of Ebola and to give a constructive suggestions regarding what strategy should be taken to deliver limited amount of drug and vaccine. Also, we planned to give an optimized plan to deliver vaccine and drug under a simplified case based on the real case of the recent outbreak in west Africa.

	We constructed models based on the biological features of EVD, social features of human society and several reasonable assumptions. Our models consist of two parts: one is considering the the spread of disease within a single city with SIR model and serves as the base of the other; the other takes the people flow among the cities into account, the application of which gave us an optimized plan regarding how should we allocate the resources of medication such as vaccine. In fact, our model is a combination of classic SIR model and graph theory, which is a simple method to solve geography related disease spread problem.

	Both of the models were applied to specific cases separately, and the results of computation which were carefully studied justified our model. Through our analysis of the model, we explored and explained the complex relationship among numerous variables and parameters. Then we find the existence of threshold values for those relationship, which indicates a limit condition for outbreak. According to the analysis and literature's instruction, we put forward our own criterion.

	The effectiveness of medical treatment (including segregation, vaccination and pharmacotherapy) is verified by our model and the strategy to allocate vaccine and drug is revealed by our investigation. Specifically, the amount of vaccine or drug delivered to each city should be roughly proportional the scale of the city and the amount of vaccine or drug allocated per capita should be larger for the cities in the center of people flow network.
\end{abstract}